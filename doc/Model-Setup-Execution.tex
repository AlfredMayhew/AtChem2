\section{Model Setup and Execution} \label{sec:setup}

AtChem2 is designed to build and run atmospheric chemistry box-models
based upon the Master Chemical Mechanism (MCM,
http://mcm.leeds.ac.uk/MCM/). This page explains how to set up, compile
and run an atmospheric chemistry box-model with AtChem2. The directory
structure of AtChem2 is described {[}{[}here\textbar{}1.2 Model
Structure{]}{]}. A working knowledge of the \textbf{unix shell} and its
\href{https://swcarpentry.github.io/shell-novice/reference/}{basic
commands} is \emph{required} to use the AtChem2 model.

There are two sets of inputs to AtChem2 - the mechanism file, and
configuration files.

\hypertarget{mechanism-file}{%
\subsection{Mechanism file}\label{mechanism-file}}

The model requires a chemical mechanism in FACSIMILE format
(\texttt{.fac}). The \textbf{mechanism file} can be downloaded from the
MCM website using the
\href{http://mcm.leeds.ac.uk/MCMv3.3.1/extract.htt}{extraction tool} or
assembled manually. The user can modify the \texttt{.fac} file as
required with a text editor. This mechanism file is converted into a
shared library and a set of associated data files in the compilation
step below. For more information on the chemical mechanism go to:
{[}{[}2.1 Chemical Mechanism{]}{]}.

\hypertarget{configuration-files}{%
\subsection{Configuration files}\label{configuration-files}}

The \textbf{model configuration} is set via a number of text files
located in the \texttt{model/configuration/} directory. The text files
can be modified with a text editor. Detailed information on the
configuration files can be found in the corresponding wiki pages:

\begin{itemize}
\item
  model and solver parameters - see {[}{[}2.2 Model Parameters{]}{]} and
  {[}{[}2.3 Solver Parameters{]}{]}.
\item
  environment variables - see {[}{[}2.4 Environment Variables{]}{]}.
\item
  photolysis rates - see {[}{[}2.5 Photolysis Rates and JFAC{]}{]}.
\item
  initial concentrations of chemical species and lists of output
  variables - see {[}{[}2.6 Config Files{]}{]}.
\end{itemize}

The model constraints - chemical species, environment variables,
photolysis rates - are located in the \texttt{model/constraints/}
directory. For more information, go to: {[}{[}2.7 Constraints{]}{]}.

\hypertarget{compilation}{%
\subsection{Compilation}\label{compilation}}

The script \texttt{build.sh} in the \texttt{tools/} directory is used to
process the chemical mechanism file (\texttt{.fac}) and to compile the
model. The script generates one Fortran file (\texttt{mechanism.f90}),
one shared library (\texttt{mechanism.so}) and four configuration files
(\texttt{mechanism.prod}, \texttt{mechanism.reac},
\texttt{mechanism.ro2}, \texttt{mechanism.species}) in the
\texttt{model/configuration/} directory. Go to the {[}{[}chemical
mechanism page \textbar{}2.1 Chemical Mechanism{]}{]} for more
information.

The script must be run from the \emph{AtChem2 main directory} and takes
four arguments (see the \textbf{Important Note 2} at the end of this
section):

\begin{enumerate}
\def\labelenumi{\arabic{enumi}.}
\item
  the path to the chemical mechanism file - no default (suggested:
  \texttt{model/}).
\item
  the path to the directory for the Fortran files generated from the
  chemical mechanism - default: \texttt{model/configuration/}.
\item
  the path to the directory with the configuration files - default:
  \texttt{model/configuration/}.
\item
  the path to the directory with the MCM data files - default:
  \texttt{mcm/}.
\end{enumerate}

For example, if the \texttt{.fac} file is in the \texttt{model/}
directory:

\begin{verbatim}
./tools/build.sh model/mechanism.fac model/configuration/ model/configuration/ mcm/
\end{verbatim}

An installation of AtChem2 can have multiple \texttt{model/}
directories, which may correspond to different models or different
projects; this allows the user to run more than one model at the same
time. In the following example, there are two \texttt{model/}
directories, each with their own chemical mechanism, configuration,
constraints and output:

\begin{verbatim}
AtChem2/
        | mcm/
        | model_1/
             | configuration/
             | constraints/
             | output/
             | mechanism.fac
        | model_2/
             | configuration/
             | constraints/
             | output/
             | mechanism.fac
        | obj/
        | src/
        | tools/
        | travis/
\end{verbatim}

Each model can be built by passing the correct path to the
\texttt{build.sh} script (see the \textbf{Important Note 1} at the end
of this section). For example:

\begin{verbatim}
./tools/build.sh model_1/mechanism.fac model_1/configuration/ model_1/configuration/ mcm/
./tools/build.sh model_2/mechanism.fac model_2/configuration/ model_2/configuration/ mcm/
\end{verbatim}

Compilation is required only once for a given \texttt{.fac} file. If the
user changes the configuration files, there is no need to recompile the
model. Likewise, if the constraints files are changed, there is no need
to recompile the model. This is because the model configuration and the
model constraints are read by the executable at runtime. However, if the
user makes changes to the \texttt{.fac} file, then the shared library
\texttt{model/configuration/mechanism.so} needs to be recompiled from
the source file \texttt{model/configuration/mechanism.f90} using the
\texttt{build.sh} script.

The user may want or need to change the Fortran code
(\texttt{src/*.f90}), in which case the model needs to be recompiled: if
the \texttt{.fac} file has also been changed, use the \texttt{build.sh}
script, as explained above. Otherwise, if only the Fortran code has been
changed, executing \texttt{make} from the \emph{main directory} is
enough to recompile the model.

\hypertarget{execution}{%
\subsection{Execution}\label{execution}}

The compilation process creates an executable file called
\texttt{atchem2} in the \emph{main directory}. The executable file takes
seven arguments, corresponding to the directories containing the model
configuration and output:

\begin{enumerate}
\def\labelenumi{\arabic{enumi}.}
\item
  the path to the directory for the model output - default:
  \texttt{model/output}
\item
  the path to the directory for the model output reaction rates -
  default: \texttt{model/output/reactionRates}
\item
  the path to the directory with the configuration files - default:
  \texttt{model/configuration/}.
\item
  the path to the directory with the MCM data files - default:
  \texttt{mcm/}.
\item
  the path to the directory with the data files of constrained chemical
  species - default: \texttt{model/constraints/species/}
\item
  the path to the directory with the data files of constrained
  environment variables - default:
  \texttt{model/constraints/environment/}
\item
  the path to the directory with the data files of constrained
  photolysis rates - default: \texttt{model/constraints/photolysis/}
\end{enumerate}

The model can be run by executing the \texttt{atchem2} command from the
\emph{main directory}, in which case the executable will use the default
configuration and output directories. Otherwise, the configuration and
output directories need to be specified (see the \textbf{Important Note
2} at the end of this section).

For example, if the constraints are in the default directories (or not
used), the model can be run by executing:

\begin{verbatim}
./atchem2 model/output/ model/output/reactionRates/ model/configuration/
\end{verbatim}

In the case of multiple \texttt{model/} directories, the directories
corresponding to each model need to be passed as arguments to the
\texttt{atchem2} executable. This allows the user to run two or more
models simultaneously. For example:

\begin{verbatim}
    ./atchem2 model_1/output/ model_1/output/reactionRates/ model_1/configuration/ mcm/ model_1/constraints/species/ model_1/constraints/environment/ model_1/constraints/photolysis/
    ./atchem2 model_2/output/ model_2/output/reactionRates/ model_2/configuration/ mcm/ model_2/constraints/species/ model_2/constraints/environment/ model_2/constraints/photolysis/
\end{verbatim}

\hypertarget{important-note-1}{%
\subsubsection{Important Note 1}\label{important-note-1}}

As explained above, if the chemical mechanism (\texttt{.fac}) is
changed, only the shared library needs to be recompiled. This allows the
user to have only one base executable called \texttt{atchem2} in the
\emph{main directory}: when running multiple models at the same time the
user can reuse this base executable while pointing each model to the
correct shared library and configuration files.

\hypertarget{important-note-2}{%
\subsubsection{Important Note 2}\label{important-note-2}}

The arguments need to be passed to the \texttt{atchem2} executable in
the exact order, as listed above. This means that if - for example - the
third argument needs to be specified, it is also necessary to specify
the first and the second arguments, even if they have the default
values. To avoid mistakes, the user can choose to always specify all the
arguments. This behaviour also applies to the \texttt{tools/build.sh}
script used to compile the model. Future versions of AtChem2 will adopt
a simpler command-line interface.

\hypertarget{output}{%
\subsection{Output}\label{output}}

The model output is saved by default in the directory
\texttt{model/output/}. The location can be modified by changing the
arguments of the \texttt{atchem2} executable (see above).

The AtChem2 output files are space-delimited text files, with a header
containing the names of the variables:

\begin{itemize}
\item
  values of environment variables and concentrations of chemical
  species: \texttt{environmentVariables.output},
  \texttt{speciesConcentrations.output}.
\item
  values of photolysis rates and related parameters:
  \texttt{photolysisRates.output},
  \texttt{photolysisRatesParameters.output}.
\item
  loss and production rates of selected species (see {[}{[}2.6 Config
  Files{]}{]}): \texttt{lossRates.output},
  \texttt{productionRates.output}.
\item
  Jacobian matrix (if requested, see {[}{[}2.2 Model Parameters{]}{]}):
  \texttt{jacobian.output}.
\item
  model diagnostic variables: \texttt{finalModelState.output},
  \texttt{initialConditionsSetting.output},
  \texttt{mainSolverParameters.output}.
\end{itemize}

In addition, the reaction rates of all the reactions in the chemical
mechanism are saved in the directory \texttt{reactionRates/}: one file
for each model step, with the filename corresponding to the time in
seconds.

\hypertarget{running-on-hpc}{%
\subsection{Running on HPC}\label{running-on-hpc}}

Atchem2 can be set up to run on a High Performance Computing (HPC)
system. Compilation and configuration are the same as for a normal
workstation. Typically, a job scheduler is used to allocate computing
resources on an HPC system. A \textbf{submission script} is therefore
needed to submit the AtChem2 models for execution.

The format and the syntax of the submission script depend on the
specific software installed on the HPC system. For instructions on how
to prepare a submission script for AtChem2, check the local
documentation or ask the HPC system administrator.

An \emph{example} submission script for the
\href{https://en.wikipedia.org/wiki/Portable_Batch_System}{Portable
Batch System (PBS)} is shown below:

\begin{verbatim}
#PBS -o atchem2.log
#PBS -e atchem2_error.log
#PBS -N base_v1
#PBS -l walltime=15:00:00
#PBS -l vmem=10gb
#PBS -m bea
#PBS -l nodes=1:ppn=1

cd ~/AtChem2/
MODELDIR="base_model_v1"
./atchem2 $MODELDIR/model/output/ $MODELDIR/model/output/reactionRates/ $MODELDIR/model/configuration/ $MODELDIR/mcm/ $MODELDIR/model/constraints/species $MODELDIR/model/environment/species $MODELDIR/photolysis/constraints/species
\end{verbatim}
