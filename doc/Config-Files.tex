\section{Config Files} \label{sec:config}

The \textbf{configuration files} contain the settings for the initial
conditions, the constraints and the output of the model. These files
complement the configuration settings of the model (in
\texttt{model.parameters}) and of the solver (in
\texttt{solver.parameters}), which are in the same directory. For more
information go to: {[}{[}2.2 Model Parameters{]}{]} and {[}{[}2.3 Solver
Parameters{]}{]}).

The configuration files have the extension \texttt{.config} and, by
default, are in the directory \texttt{model/configuration/}. This
directory also contains the files generated during the {[}{[}build
process\textbar{}2. Model Setup and Execution{]}{]} which describe the
chemical mechanism (\texttt{mechanism.species}, \texttt{mechanism.reac},
\texttt{mechanism.prod}, \texttt{mechanism.ro2}), as explained in the
{[}{[}chemical mechanism page\textbar{}2.1 Chemical Mechanism{]}{]}. The
location of the configuration files can be modified by changing the
arguments of the script \texttt{tools/build.sh} (see {[}{[}2. Model
Setup and Execution{]}{]}).

The content and the format of the \texttt{.config} files are described
below. Note that the names of some files have changed with the release
of \textbf{version 1.1} (November 2018).

\hypertarget{environmentvariables.config}{%
\subsection{environmentVariables.config}\label{environmentvariables.config}}

This file contains the settings for the environment variables, which are
described in detail in the related {[}{[}wiki page\textbar{}2.4
Environment Variables{]}{]}. If an environment variable is constrained,
there must be a corresponding data file in
\texttt{model/constraints/environment/} (see {[}{[}2.7
Constraints{]}{]}).

\hypertarget{initialconcentrations.config}{%
\subsection{initialConcentrations.config}\label{initialconcentrations.config}}

This file contains the initial concentrations of the chemical species.
The first column is the list of initialized species, the second column
is the corresponding concentration at \texttt{t\ =\ 0} (in
\textbf{molecules cm-3}). For example:

\begin{verbatim}
NO      378473308.14
NO2     86893908168.9
O3      1.213e+12
CH4     4.938e+13
\end{verbatim}

The chemical species not included in this file are automatically
initialized to the default value \texttt{0}. It is not necessary to
initialize the constant and the constrained species (i.e., those listed
in \texttt{speciesConstant.config} and
\texttt{speciesConstrained.config}).

The environment variables are set in
\texttt{environmentVariables.config} (see above) and should not be
included in this file.

\hypertarget{outputrates.config}{%
\subsection{outputRates.config}\label{outputrates.config}}

This file (called \texttt{productionRatesOutput.config} and
\texttt{lossRatesOutput.config} in v1.0) lists the chemical species for
which detailed production rates and loss rates are required. The file
has one column, with one species per line.

The frequency of this output is controlled by the \textbf{rates output
step size} parameter in \texttt{model.parameters} (see {[}{[}2.2 Model
Parameters{]}{]}). The format of the corresponding output files -
\texttt{lossRates.output} and \texttt{productionRates.output} - is
designed to facilitate the analysis of production and destruction rates
for selected species of interests (rather than processing the output
files in \texttt{model/output/reactionRates/}):

\begin{verbatim}
     time        speciesNumber    speciesName    reactionNumber         rate        reaction

3.600000E+003           8               OH              15         0.000000E+000    O1D=OH+OH
3.600000E+003           8               OH              20         0.000000E+000    HO2+O3=OH
3.600000E+003           9               HO2             16         0.000000E+000    OH+O3=HO2
3.600000E+003           9               HO2             17         0.000000E+000    OH+H2=HO2

7.200000E+003           8               OH              15         0.000000E+000    O1D=OH+OH
7.200000E+003           8               OH              20         0.000000E+000    HO2+O3=OH
7.200000E+003           9               HO2             16         0.000000E+000    OH+O3=HO2
7.200000E+003           9               HO2             17         0.000000E+000    OH+H2=HO2
\end{verbatim}

\hypertarget{outputspecies.config}{%
\subsection{outputSpecies.config}\label{outputspecies.config}}

This file (called \texttt{concentrationOutput.config} in v1.0) lists the
chemical species for which the model output is required. The current
version of AtChem2 limits the number of species that can be output to
100, although the user can modify the Fortran code to increase this
number. The file has one column, with one species per line.

The frequency of this output is controlled by the \textbf{step size}
parameter in \texttt{model.parameters} (see {[}{[}2.2 Model
Parameters{]}{]}).

\hypertarget{photolysisconstant.config}{%
\subsection{photolysisConstant.config}\label{photolysisconstant.config}}

This file lists the photolysis rates that are constant. The file has
three columns: the first column is the number that identifies the
photolysis rate (e.g., \texttt{1}), the second column is the value of
the photolysis rate in \textbf{s-1} (e.g., \texttt{1e-5}), the third
column is the name of the photolysis rate (e.g., \texttt{J1}). The
photolysis rates are named according to the
\href{http://mcm.leeds.ac.uk/MCMv3.3.1/parameters/photolysis.htt}{MCM
naming convention}. If no photolysis rate is constant, the file should
be left empty.

If one or more photolysis rates is set to a constant value, the others
(i.e., those not listed in \texttt{photolysisConstants.config}) are set
to zero. For more information go to: {[}{[}2.5 Photolysis Rates and
JFAC{]}{]}.

\hypertarget{photolysisconstrained.config}{%
\subsection{photolysisConstrained.config}\label{photolysisconstrained.config}}

This file (called \texttt{constrainedPhotoRates.config} in v1.0) lists
the photolysis rates that are constrained. The file has one column, with
one photolysis rate per line (e.g., \texttt{J1}). The photolysis rates
are named according to the
\href{http://mcm.leeds.ac.uk/MCMv3.3.1/parameters/photolysis.htt}{MCM
naming convention}. If no photolysis rate is constrained, the file
should be left empty. If a photolysis rate is constrained, there must be
a corresponding data file in \texttt{model/constraints/photolysis/} (see
{[}{[}2.7 Constraints{]}{]}).

The photolysis rates that are not listed in
\texttt{photolysisConstrained.config} are calculated by AtChem2 using
the MCM parametrization and the parameters in
\texttt{mcm/photolysis-rates\_v3.3.1}. Older versions of the MCM
photolysis parametrization can be used, as explained in the file
\texttt{mcm/INFO.md}. For more information go to: {[}{[}2.5 Photolysis
Rates and JFAC{]}{]}.

\hypertarget{speciesconstant.config}{%
\subsection{speciesConstant.config}\label{speciesconstant.config}}

This file (called \texttt{constrainedFixedSpecies.config} in v1.0) lists
the chemical species that are constant. The file has two columns: the
first column is the list of constant species, the second column is the
corresponding concentration (in \textbf{molecules cm-3}). If no chemical
species is constant, the file should be left empty.

If a chemical species is constant, it does not need to be initialized:
the values set in \texttt{speciesConstant.config} override those set in
\texttt{initialConcentrations.config}.

\hypertarget{speciesconstrained.config}{%
\subsection{speciesConstrained.config}\label{speciesconstrained.config}}

This file (called \texttt{constrainedSpecies.config} in v1.0) lists the
chemical species that are constrained. The file has one column, with one
species per line. If no chemical species is constrained, the file should
be left empty. If a chemical species is constrained, there must be a
corresponding data file in \texttt{model/constraints/species/} (see
{[}{[}2.7 Constraints{]}{]}).

If a chemical species constrained, it does not need to be initialized:
the values set in \texttt{speciesConstrained.config} override those set
in \texttt{initialConcentrations.config}.
